\section{Περιληπτικά}

Η στεφανιαία καρδιακή νόσος, ή απλά στεφανιαία νόσος, είναι μια ασθένεια της καρδιάς. Προκαλείται από το φράξιμο των αθηρωματικών αρτηριών, με συνέπεια να εμποδίζεται η παροχή αίματος στην καρδιά. Αυτό, έχει ως αποτέλεσμα τη μειωμένη παροχή οξυγόνου και θρεπτικών ουσιών στους ιστούς της καρδιάς.

Η στεφανιαία νόσος είναι η πρωταρχική αιτία θανάτου στις σύγχρονες κοινωνίες, με ποσοστό 15.6\% επί των συνολικών θανάτων παγκοσμίως. Το 2015, επηρέασε 110 εκατομμύρια ανθρώπους, και προκάλεσε 8.9 εκατομμύρια θανάτους \cite{wikiCAD}.

Συνηθέστερη αιτία πρόκλησης της στεφανιαίας νόσου είναι η αθηροσκλήρωση, κατά την οποία δημιουργούνται αθηρωματικές πλάκες, οι οποίες επικάθονται στον αυλό των στεφανιαίων αρτηριών, δυσκολεύοντας έτσι τη φυσιολογική ροή του αίματος. Οι αθηρωματικές πλάκες σχηματίζονται από εναποθέσεις λίπους στις αρτηρίες, μια κατάσταση που ονομάζεται αρτηριοσκλήρυνση, και οδηγεί στην στένωση ή την απόφραξη των αγγείων. Η ρίξη της αθηρωματικής πλάκας ονομάζεται αθηροθρόμβωση και ακολουθείται από πλήρη και παρατεταμένη έλλειψη οξυγόνου στο μυοκάρδιο, που με τη σειρά του προκαλεί τη νέκρωση του μυοκαρδίου, γνωστή και ως έμφραγμα.

Παράγοντες οι οποίοι συνδέονται με αυξημένο κίνδυνο πρόκλησης στεφανιαίας νόσου είναι η υψηλή αρτηριακή πίεση, το κάπνισμα, ο διαβήτης, η έλλειψη άσκησης, η παχυσαρκία, η κατάθλιψη και η υπερβολική κατανάλωση αλκοόλ.

Οι συνηθέστεροι τρόποι διάγνωσης της στεφανιαίας νόσου είναι η στεφανιογραφία και το ηλεκτροκαρδιογράφημα. Η διάγνωση της στεφανιαίας νόσου κρίνεται ιδιαίτερα σημαντική, καθώς με τις κατάλληλες αλλαγές στον τρόπο ζωής, μπορούν να προληφθούν σε σημαντικό ποσοστό τα επικίνδυνα συμπτώματά της. \cite{abstract1, abstract2}

Εξαιτίας της σοβαρότητας αυτής της νόσου, τα τελευταία χρόνια έχουν γίνει πολλές προσπάθειες στον τομέα της βιοϊατρικής για να δημιουργηθούν αυτοματοποιημένοι και εύκολοι τρόποι διάγνωσης. Ανάμεσα σε αυτές, είναι και ορισμένοι έξυπνοι αλγόριθμοι τεχνητής νοημοσύνης (αναφορές θέλει εδώ), οι οποίοι χρησιμοποιούν συνήθως κάποιο ηλεκτροκαρδιογράφημα του ασθενή και προσπαθούν να διαγνώσουν εάν ο ασθενής έχει στεφανιαία νόσο. Ωστόσο, ως επί τω πλείστω, οι αλγόριθμοι αυτοί χρειάζονται ως είσοδο 
ένα ηλεκτροκαρδιογράφημα μεγάλης χρονικής διάρκειας (συνήθως 24ωρο), δυσκολεύοντας έτσι τη χρήση τους στην καθημερίνη ζωή.

Η παρούσα μελέτη, έχει ως σκοπό τη δημιουργία μιας ολοκληρωμένης εφαρμογής στο λειτουργικό σύστημα \selectlanguage{english}Apple iOS\selectlanguage{greek}, στην οποία ο χρήστης θα μπορεί να λαμβάνει ένα ή περισσότερα καρδιογραφήματα διάρκειας 30 δευτερολέπτων, και ο αλγόριθμος τεχνητής νοημοσύνης που έχει κατασκευαστεί να μπορεί να εμφανίζει την πιθανότητα εμφάνισης στεφανιαίας νόσου στον ασθενή.